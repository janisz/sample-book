% This is a simplified version of the Softcover stylesheet


\makeatletter % changes the catcode of @ to 11


%% Page size
%% These settings are optimized for ebooks.
%% If you want different settings, override them in custom_pdf.sty.
%\setlength{\oddsidemargin}{.25in}
%\setlength{\evensidemargin}{.25in}
%\setlength{\textheight}{8in}
%\setlength{\textwidth}{6.25in}
%\setlength{\topmargin}{0in}

%% Font encodings
%\usepackage[T1]{fontenc}
%% Be able to define colors
\usepackage[svgnames]{xcolor}
%% Be able to include book covers
%\usepackage{pdfpages}
%% Include graphics
%\usepackage{graphicx}
\def\maxwidth#1{\ifdim\Gin@nat@width>#1 #1\else\Gin@nat@width\fi}
%% Support the \url command
%\usepackage{url}


% Commands included by PolyTeXnic
\usepackage{latex_styles/polytexnic_commands}






%% Links
\definecolor{darkblue}{rgb}{0,0.18,0.45}
\definecolor{darkgreen}{rgb}{0,0.39,0}
%% Bizarrely, we need to define the ALL CAPS version of DARKGREEN to account
%% for some edge cases, whose nature remains mysterious.
\definecolor{DARKGREEN}{rgb}{0,0.39,0}
\definecolor{hilightyellow}{rgb}{1.0,1.0,0.8}
%% Configure hyperref footnotes
%\usepackage[hyperfootnotes=false]{hyperref}
%\hypersetup{hyperfootnotes=false}
%\hypersetup{colorlinks,linkcolor=darkblue,urlcolor=blue}
%% Syntax highlighting


\usepackage{latex_styles/pygments}
% Change color of '@go', "general output", from gray to dark green.
\expandafter\def\csname PY@tok@go\endcsname{\def\PY@tc##1{\textcolor{darkgreen}{##1}}}



%% American Mathematical Society extensions
%\usepackage{amsmath}
%\usepackage{amsfonts}

% 'Verbatim' environment
\usepackage{fancyvrb}

%% Be able to resize text relative to other text
\usepackage{relsize}


% Support longtable environment.
\usepackage{longtable}


% Support strikethrough (via \sout{text})
\usepackage[normalem]{ulem}

%% Configure fonts
%\renewcommand{\rmdefault}{ptm}
%\usepackage{courier}
%\normalfont % in case the EC fonts aren't available


% Code environments
\DefineVerbatimEnvironment%
  {code}{Verbatim}{fontsize=\relsize{-2.5},fontseries=b}
% The metacode environment exists solely to allow meta-discussion of the code
% environment, as in
%   %= lang:latex
%   \begin{metacode}
%   %= lang:ruby
%   \begin{code}
%   def foo
%     "bar"
%   end
%   \end{code}
%   \end{metacode}
\DefineVerbatimEnvironment%
  {metacode}{Verbatim}{fontsize=\relsize{-2.5},fontseries=b}
% Use a nice font in code environments.
\usepackage[scaled=0.92]{helvet}

% Filesystem paths
\newcommand{\filepath}[1]{\textit{\texttt{\small #1}}}

% Size-constrained images
\newcommand{\image}[1]{\begin{center}\includegraphics[width=\maxwidth{0.95\textwidth}]{#1}\end{center}}
\newcommand{\imagebox}[1]{\begin{center}\fbox{\includegraphics[width=\maxwidth{0.95\textwidth}]{#1}}\end{center}}

\newenvironment{framed_shaded}{%
  \def\FrameCommand##1{\hskip\@totalleftmargin
  \fcolorbox{boxcolor}{shadecolor}{##1}%
      % There is no \@totalrightmargin, so:
      \hskip-\linewidth \hskip-\@totalleftmargin \hskip\columnwidth}%
  \MakeFramed {\advance\hsize-\width
    \@totalleftmargin\z@ \linewidth\hsize
    \advance\labelsep\fboxsep
    \@setminipage\vspace{0.3em}}%
 }{\vspace{-0.6em}\par\unskip\@minipagefalse\endMakeFramed}

\newenvironment{full_framed_shaded}{%
  \def\FrameCommand{\fboxsep=\FrameSep\fcolorbox{boxcolor}{shadecolor}}%
  \MakeFramed {\advance\hsize\width \FrameRestore}}%
 {\endMakeFramed}

\newenvironment{container}{%
  \def\FrameCommand##1{\hskip\@totalleftmargin \hskip-\fboxsep
  \colorbox{white}{##1}\hskip-\fboxsep
      % There is no \@totalrightmargin, so:
      \hskip-\linewidth \hskip-\@totalleftmargin \hskip\columnwidth}%
  \MakeFramed {\advance\hsize-\width
    \@totalleftmargin\z@ \linewidth\hsize
    \@setminipage}%
 }{\par\unskip\@minipagefalse\endMakeFramed}

% Caption styling
\usepackage[font={it,small}]{caption}

% Codelistings
\newcounter{codelisting}
\@ifundefined{chapter}{}{\numberwithin{codelisting}{chapter}}
\newenvironment{codelisting}{\refstepcounter{codelisting}\begin{framed_shaded}\vspace{-0.5em}}%
{\end{framed_shaded}}
% See latex_styles/language_customization.sty for codelisting captions.

% Aside boxes
\usepackage{amsthm}
\theoremstyle{definition}
\newcommand{\boxlabel}{Box}
\@ifundefined{chapter}
  {\newtheorem{aside}{\boxlabel}}
  {\newtheorem{aside}{\boxlabel}[chapter]}
\usepackage{latex_styles/framed}
\definecolor{shadecolor}{gray}{0.97}
\definecolor{boxcolor}{gray}{0.10}
\newenvironment{shaded_aside}[2]{\begin{full_framed_shaded}\begin{aside}\label{#2} \textbf{#1}\end{aside}}{\bigskip\end{full_framed_shaded}}

% Additional commands
\newcommand{\heading}[1]{\textbf{#1}}
\newcommand{\kodesize}{\smaller[0.75]}
\newcommand{\kode}[1]{\textcolor{darkgreen}{\textbf{\texttt{\kodesize #1}}}}
\newcommand{\coloredtext}[2]{\textcolor{#1}{#2}}
\newcommand{\coloredtexthtml}[2]{\textcolor[HTML]{#1}{#2}}

% Subtitle command
\usepackage{titling}
\newcommand{\subtitle}[1]{%
  \posttitle{%
    \par\end{center}
    \begin{center}\large#1\end{center}
    \vskip0.5em}%
}

% Enable the \pbox command for paragraph boxes in tables.
\usepackage{pbox}

%% Enable float placement options.
%\usepackage{float}

%
%% Define some commonly used Unicode characters.
%\usepackage{latex_styles/applekeys}
%\usepackage{newunicodechar}
%\newunicodechar{⌘}{\cmdkey}
%\newunicodechar{⌥}{\optkey}
%\newunicodechar{⌃}{\ctlkey}
%\newunicodechar{⇧}{\shiftkey}
%\newunicodechar{→}{\ensuremath{\rightarrow}}
%\newunicodechar{←}{\ensuremath{\leftarrow}}
%\newunicodechar{↑}{\ensuremath{\uparrow}}
%\newunicodechar{↓}{\ensuremath{\downarrow}}
%\newunicodechar{⇥}{\tabkey}
%\newunicodechar{↵}{\returnkey}
%\newunicodechar{⌫}{\delkey}
%\newunicodechar{␣}{\textvisiblespace}
%\newunicodechar{—}{\textemdash}
%\newunicodechar{–}{\textendash}
%\newunicodechar{™}{\texttrademark}
%\newunicodechar{©}{\copyright}
%\newunicodechar{®}{\textregistered}
%\newunicodechar{…}{\ldots}
%\newunicodechar{£}{\pounds}
%\usepackage{eurosym}
%\newunicodechar{€}{\euro}
%\newunicodechar{¡}{!`}
%\newunicodechar{¿}{?`}
%\newunicodechar{ß}{\ss}
%\newunicodechar{✓}{\checkmark}
%
%% xelatex supports macrons by default, but for some reason they disappear.
%\newunicodechar{ā}{\={a}}
%\newunicodechar{ē}{\={e}}
%\newunicodechar{ī}{\={\i}}
%\newunicodechar{ō}{\={o}}
%\newunicodechar{ū}{\={u}}
%\newunicodechar{Ā}{\={A}}
%\newunicodechar{Ē}{\={E}}
%\newunicodechar{Ī}{\={I}}
%\newunicodechar{Ō}{\={O}}
%\newunicodechar{Ū}{\={U}}
%\newunicodechar{«}{\guillemotleft}
%\newunicodechar{»}{\guillemotright}


\usepackage{latex_styles/language_customization}


% Fix quotes in code environments.
% Provides "upquote.sty" functionality compatible with the latest Pygments.
%\RequirePackage{textcomp}
\begingroup
\catcode`'=\active
\catcode``=\active
\g@addto@macro\@noligs
   {\let`\textasciigrave
    \let'\textquotesingle
    \let\PYZsq\textquotesingle}
\endgroup

% Add smallcaps
%\usepackage{fontspec}
%\setmainfont[ItalicFont     = Times New Roman Italic,
%             BoldFont       = Times New Roman Bold,
%             BoldItalicFont = Times New Roman Bold Italic,
%             SmallCapsFont  = Bodoni 72 Smallcaps]
%            {Times New Roman}
%\setmonofont{Courier}

% Include custom commands.
\usepackage{latex_styles/custom}
\usepackage{latex_styles/custom_pdf}

\makeatother % changes the catcode of @ back to 12
