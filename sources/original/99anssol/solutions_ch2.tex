\begin{Solution}{2.1}
			The biased coin flip is modelled by a random variable $Y$,
			and different coin flips correspond to random variables $Y_1$, $Y_2$, $Y_3$, \ldots which are independent copies of $Y$.
			The probability of getting \texttt{heads} on the first flip is $P_N(1)=\textrm{Pr}\!\left( \{ Y_1=\texttt{heads} \} \right)\! =p$.
			The probability of getting \texttt{heads} on the second flip corresponds
			to the event $\{Y_1=\texttt{tails}\} \ \texttt{AND} \ \{Y_2=\texttt{heads} \}$.
			We assumed the coin flips are independent so % the probability of this event is the product
			$P_N(2)=(1-p)p$.
			Similarly $P_N(3) = (1-p)^2p$.
			The general formula is $P_N(n) = (1-p)^{n-1}p$.
		
\end{Solution}
